\documentclass[oneside]{article}   	% use "amsart" instead of "article" for AMSLaTeX format
\usepackage{geometry}                		% See geometry.pdf to learn the layout options. There are lots.
\usepackage{graphicx}				% Use pdf, png, jpg, or eps§ with pdflatex; use eps in DVI mode
								% TeX will automatically convert eps --> pdf in pdflatex		
\usepackage{amssymb}

%SetFonts

%SetFonts


\title{Heuristic Analysis}
\author{Ilya Nikokoshev}
\date{January 1, 2018}

\begin{document}
\maketitle

\begin{abstract}
As part of the Project 2 in the Artificial Intelligence Nanodegree Program, we implement three heuristics for score computation.
\end{abstract}

\tableofcontents

\section{Heuristic Descriptions}

All of our heuristics attempt to express the goodness of a position as a value that would be larger whenever there seem to be more choices for the first player (denoted $A$) and fewer choices for the opponent (denoted $B$).


\subsection{Distance Heuristic $D(z, \alpha, f)$}

\subsubsection{Definition}

For each blank space $s$ on the board, we find whether it is accessible from the current positions of players $A$ and $B$. If it is not accessible by either, this square does not contribute to the heuristic value. 

Otherwise, let $d^A_s$ and $d^B_s$ denote the minimum number of steps required to move to this square from the positions of, respectively, player $A$ and $B$, or $+\infty$ if this square is inaccessible. We set the contribution of $s$ to the heuristic then as $\alpha f(d^B_s-d^A_s)$,
where $f$ is a suitably chosen monotonic function with the property $P$ that $f(+\infty) = 1$ and $f(-\infty) = -1$. 

The total value of the heuristic will be therefore computed as 

$$D(z, \alpha, f) = z + \alpha\sum_{s\in A}f(d^B_s-d^A_s), $$

where $A$ is the set of accessible spaces on the board, $d^P_s$ is defined above, and $z$ provides the initial value.

\subsubsection{Theoretical analysis}
\label{DistanceAnalysis}

We will select either $\arctan$ or sigmoid function $(1+e^{-x})^{-1}$ as the function $f$ (both of them, of course, suitably rescaled). 

The condition $P$ ensures that if the board is divided into two disjoint parts, so that $A$ only has access to $n_A$ cells, and $B$ only has access to some other $n_B$ cells, the value of the heuristic $D = z+\alpha(n_A-n_B)$. Naively, such a position is likely to be winning for $A$ if and only if $n_A > n_B$. Moreover, for the cases when $n_A=n_B=0$ ($A$ loses immediately), and $n_A=1, n_B=0$ ($B$ loses after $A$'s move), the utility is known to be equal to $-1$ and $1$. If we want the value of $D$ to be equal to the utility in these two cases, we should select $\alpha = 2, z = -1/2$.

\subsubsection{Validating the parameter choices}
\label{distancechoices}

For comparison, we provide results of the competition of matches between selected distance heuristics with different values of parameters and \texttt{improved\_score} heuristic in Table \ref{DistanceCompetition}. In the table, cells corresponds to the percentage of wins of an alphabeta search agent with a row heuristic against the column heuristic in 100 matches, and the average is computed from rows as well as columns (thus, it is an average win in 700 parties against other opponents).

\begin{table}[htp]
\caption{Distance heuristics competition}
\begin{center}
\begin{tabular}{c|ccc|rcccc|c}
   & $z$ & $\alpha$ & $f$ & Wins against 1 & 2 & 3 & 4 & Improved & Average \\
   \hline
 Variant 1 &    0 & 1 & $\arctan$  & $52\%$ & $52\%$ & $51\%$ & $58\%$ & $57\%$ & $53.0\%$ \\
 Variant 2 & -1/2 & 1 & $\arctan$  & $50\%$ & $52\%$ & $51\%$ & $47\%$ & $45\%$ & $49.3\%$ \\
 Variant 3 & -1/2 & 2 & $\arctan$  & $48\%$ & $51\%$ & $55\%$ & $48\%$ & $51\%$ & $48.9\%$ \\
 Variant 4 & -1/2 & 2 & sigmoid    & $49\%$ & $45\%$ & $54\%$ & $49\%$ & $49\%$ & $49.1\%$
 
\end{tabular}
\end{center}
\label{DistanceCompetition}
\end{table}%

The table suggests that, contrary to the theoretical analysis in \ref{DistanceAnalysis}, the Variant 1 heuristic is the strongest, and 3 and 4 are the weakest.

Unfortunately, the standard deviation for the case of tossing a perfect coin 700 times comprises about 2\%, which means that the results in the average column cannot rule out the hypothesis that all of the opponents have equal strength. Thus the confidence in the results above is not very high.

In an attempt to further examine the question of parameters, we put Variant 1 against Variant 4 in two other competitions with 500 matches, not only on a standard $7\times 7$, but also on a $13\times 13$ boards. The results, presented in Table \ref{DistanceBoards}, do not strongly support selecting one choice over the other. 

\begin{table}[htp]
\caption{Further comparison of distance heuristic variants 1 and 4}
\begin{center}
\begin{tabular}{c|c}
   Board size & Wins of 4 over 1 \\
   \hline
 $7\times 7$   & $52.6\%$ \\
 $13\times 13$ & $50.6\%$
 
\end{tabular}
\end{center}
\label{DistanceBoards}
\end{table}%

In combination with the theoretical analysis, we therefore make a choice to take both $D(0, 1, \arctan)$ and $D(-1/2, 2, \textrm{sigmoid})$ as the preferred distance heuristics.


\subsection{Free Moves Heuristic $F(d, u, \beta)$}


\section{Final Choices}


We combine the results of \ref{distancechoices} to select three heuristic functions and present the  results of a competition between them and \texttt{improved\_score} heuristics in Table \ref{FinalCompetition}.

\begin{table}[htp]
\caption{Final heuristics competition}
\begin{center}
\begin{tabular}{clc|cccc}
   & Function & Heuristic & 1 & 2 & 3 & Improved \\
   \hline
1 & \texttt{custom\_score}    & $D(0, 1, \arctan)$             & & & &  \\
2 & \texttt{custom\_score\_2} & $D(-1/2, 2, \textrm{sigmoid})$ & & & & \\
3 & \texttt{custom\_score\_3} & ?? & & & &
\end{tabular}
\end{center}
\label{FinalCompetition}
\end{table}%



\end{document}  